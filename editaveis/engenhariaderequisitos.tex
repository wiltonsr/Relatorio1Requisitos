\chapter[Engenharia de Requisitos]{Engenharia de Requisitos}

\section{Scaled Agile Framework}
\section{Nível de Portfólio}
\section{Nível de Programa}
Esse nível tem como principais objetivos manter a visão (Documento de Visão), gerenciar a release, gerenciar a qualidade (integrando os resultados obtidos pelos times, e que os padrões de qualidade, performance, entre outros, estão sendo assegurados), fazer o deploy do sistema, gerenciar recursos (ajustando prazo e gastos), e eliminarem possíveis impedimentos (serão os facilitadores) \cite{safe002}.



\section{Nível de Time}
A unidade básica de trabalho para este nível é a História de Usuário. O objetivo deste nível é definir, construir e testar as histórias de usuário no escopo da iteração, afim de se concluir mais partes do produto final \cite{safe001}.

\section{Papéis}
Um papel define comportamentos e responsabilidades de um indivíduo ou de um grupo de indivíduos que trabalham juntos como um time \cite{kruchten002}. O corportamento é expresso em termos de atividades que o papel pratica, e, cada papel é associado a um conjunto de atividades. No SAFe existem diferentes papéis para os diferentes níveis do sistema. No Nível de Portfólio vão haver, principalmente, papéis que interagem com os épicos, no Nível de Programa, papéis que interagem com as Features, e, no Nível de Time, papéis que interagem com Histórias de Usuário.

\subsection{Nível de Portfólio}
\subsubsection{Epic Owner}
\subsubsection{Enterprise Architect}

\subsubsection{Nível de Programa}
\subsubsection{Product Management}
\subsubsection{Release Management}
\subsubsection{System Team}
\subsubsection{Business Owners}
\subsubsection{RTE}
\subsubsection{System Architect}
\subsubsection{UX}
\subsubsection{Shared Resources}

\subsection{Nível de Time}
\subsubsection{Product Owner}
\subsubsection{Scrum Master}
\subsubsection{Desenvolvedores e Testadores}
