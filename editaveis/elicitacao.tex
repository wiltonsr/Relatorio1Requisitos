\chapter[Técnicas de Elicitação de Requisitos]{Técnicas de Elicitação de Requisitos}
Em determinados pontos do ciclo de vida de um projeto, a organização precisará obter uma certa compreensão das condições que o sistema deverá satisfazer, questionamentos como "Quais problemas o sistema pretende solucionar?", "Qual público alvo ele pretende servir?", "Quais suas restrições?" irão emergir e serão respondidos pela definição dos requisitos realizada pela equipe.

Este processo de buscar respostas para as questões levantadas é conhecido como descoberta de requisitos. A equipe procura formas de identificar e descrever as características e capacidades do \emph{software} que se propõe a solucionar um problema e para isso é necessário que o time foque em manter a concordância, compreensão e entendimento entre todos os membros para que todos compartilhem uma visão do sistema que será desenvolvido.

Descobrir requisitos é um dos grandes desafios do processo de produção de Software. Não ser capaz de completar essa tarefa da maneira certa e com qualidade impossibilitará o projeto de obter sucesso, não importa quão bom sejam as qualidades do time \cite[p. 227-228]{safe001}.

Considerando a importância da descoberta de requisitos, várias técnicas são propostas para auxiliar e otimizar o processo de elicitação. Cada uma delas possui suas vantagens e desvantagens e uma combinação delas pode ser necessária para atender as circunstâncias de um determinado projeto. Neste documento, serão apresentadas sete técnicas, sendo três destas escolhidas pela equipe para serem utilizadas no projeto:

\section{Workshop de Requisitos}
O principal objetivo desta técnica é estabelecer um consenso entre os requisitos do sistema em um curto espaço de tempo. Para isso, ele indica que os principais \emph{stakeholders} do projeto devem ser reunidos com os integrantes da equipe que participarão do \emph{workshop} (Product Owner, Product Manager, um membro do time, etc...). O \emph{workshop} procura manter uma concordância à respeito dos requisitos entre os envolvidos no projeto através da exposição de ideias de todos os participantes, ele geralmente dura um dia e envolve uma introdução para informar aos participantes como o \emph{workshop} funcionará, uma contextualização para descrever o estado atual do projeto, um \emph{brainstorming} para que todos os participantes possam expor suas ideias e discutir à respeito do projeto, um almoço para servir de intervalo (apesar de que durante o almoço é recomendado que a discussão mantenha de forma saudável para que o clima do workshop não se perca), outro \emph{brainstorming} após o almoço, definição de \emph{features} para identificar requisitos de alto nível do sistema, priorização de algumas das \emph{features} propostas e para encerrar, um resumo sobre tudo que ocorreu durante o dia.

Esta técnica não será utilizada pois o esforço e tempo exigido é muito grande e não seria possível, e, além disso, é uma técnica relativamente massante.

\section{Brainstorming}
Esta técnica procura oferecer uma oportunidade para que todos os participantes exponham suas ideias em relação ao que imaginam ser uma solução potencial para o projeto. O Brainstorming é dividido em duas fases: geração de ideias e redução de ideias. Durante a geração de ideias, os participantes identificam a maior quantidade de ideias possíveis, sendo que estas propõe possíveis soluções, capacidades, restrições e características do sistema. Durante a redução de ideias, os participantes realizam a análise das ideias propostas, isso inclui um refinamento, uma priorização, uma organização, uma triagem, etc.
A principal proposta do Brainstorming é criar um ambiente que seja capaz de receber discussões saudáveis, ele incentiva que os participantes se abram e falem o que pensam e não permite críticas destrutivas já que isso desmotivaria os participantes e os colocariam em uma posição desconfortável. Esta técnica valoriza a criatividade e aposta na combinação das ideias geradas pelos participantes.

Essa técnica será utilizada pois é efetiva e flexível, que é o perfil de técnica que melhor se adequa ao time.

\section{Entrevistas e Questionários}
Esta técnica consiste na realização de diversos questionários durante uma entrevista com um \emph{stakeholder}. É sugerido que o entrevistador responsável por interagir com o cliente priorize perguntas livres de contexto primeiramente, para evitar a realização perguntas tendenciosas que possam influenciar a resposta do cliente. Após a realização das perguntas livres de contexto, é recomendado que se faça perguntas que explorem possíveis soluções, para isso, é interessante que o entrevistador compreenda o contexto do projeto, o que pode exigir um estudo.
Após entrevistar os principais \emph{stakeholders} do projeto, as respostas serão analisadas e irão produzir uma certa convergência à respeito do entendimento do problema.

\section{Mock-Ups}
A proposta desta técnica é construir um protótipo da interface de usuário referente à um requisito, para que o cliente colabore com a extração das funcionalidades contidas naquela interface. Isso permite que o cliente demonstre as possíveis interações imaginadas com o sistema antes que qualquer código seja implementado. É possível desenhar no papel a interface imaginada ou utilizar ferramentas que automatizem o processo e construam \emph{wireframes}.

\section{Análise Competitiva}
A análise competitiva é uma técnica que procura identificar possíveis concorrentes e estudar o que eles oferecem em seus produtos. À partir deste estudo, a equipe realiza uma avaliação para determinar funcionalidades potenciais do concorrente que podem ser incluídas no projeto por refletirem o interesse dos clientes.

\section{Sistema de Solicitação de Mudanças do Cliente}
Esta técnica consiste na implementação de um sistema que aceite sugestões dos usuários à respeito de possíveis evoluções e correções no \emph{software}. É basicamente um sistema de \emph{feedbacks} da experiência obtida por um usuário ao utilizar o \emph{software}.

\section{Modelagem caso-de-uso}
Apesar das equipes ágeis utilizarem primariamente estórias de usuário para descrever os requisitos de um sistema, em muitas vezes, esta abordagem se torna inadequada devido à complexidade do \emph{software}. Em situações em que o \emph{software} é composto por vários sistemas e componentes, o relacionamento entre eles pode se tornar difícil de descrever em estórias de usuário e os casos de uso podem ser mais adequados para especificar a interação entre o \emph{software} e os seus componentes (sub-sistemas, dispositivos de hardware, entre outros).

\section{Escolha das Tecnicas}
Assim sendo, a equipe optou pelas seguintes técnicas de elicitação:
\begin{itemize}
  \item Mock-ups: Imaginamos que a construção de protótipos que representem a estrutura das interfaces de usuário da nossa aplicação nos ajude a extrair, com a colaboração do cliente, as funcionalidades presentes naquele ponto do \emph{software}, já que o cliente indicará as possíveis interações que pretende ter com aquela parte do sistema.

  \item Brainstorming: Avaliamos ser de fundamental importância a troca de ideias entre os envolvidos no projeto, para que toda a informação resultante do que foi imaginado por cada um convirja para um conjunto de características e capacidades do sistema que reflitam as necessidades do cliente.

  \item Análise Competitiva: Consideramos a análise de soluções oferecidas por concorrentes de sucesso uma excelente abordagem para avaliar funcionalidades que possam ser incluídas ao nosso projeto. Soluções de sucesso que resolvem problemas parecidos com o nosso e oferecem um produto relativamente próximo à solução imaginada pode ser uma fonte de ideias para extrair as capacidades do sistema a ser desenvolvido.
\end{itemize}

