\chapter[Modelos de Maturidade]{Modelos de Maturidade}
Modelos de Maturidade, nos dias de hoje, ajudam organizações a melhorarem a maneira com que elas fazem negócio. No contexto de \emph{Engenharia de Software}, estes modelos ajudam organizações a desenvolver e manter a qualidade dos seus produtos e serviços, trazendo um conjunto de boas práticas a serem seguidas para se atingir determinado nível de maturidade \cite{cmmi001}.

No contexto deste projeto, os modelos de maturidade (em especial o CMMI e o MPS.BR) foram utilizados como forma de apoio durante no projeto. Entre essas formas, a principal foi na definição de tarefas que seriam utilizadas no processo de Engenharia de Requisitos. Esperava-se que pelo menos as áreas de processo relacionadas a área de requisitos (\emph{REQM - Requirements Management, e RD - Requirements Development}) tivessem todas as suas práticas presentes no projeto, direta ou indiretamente, então, foi feita essa relação entre o modelo de maturidade, a abordagem escolhida (adaptativa - SAFe) e o projeto em si.